%%%%%%%%%%%%%%%%%%%%%%%%%%%%%%%%%%%%%%%%%%%%%%%%%%%%%%%%%%%%%%%%%%%%%%%%%%%%%%
% Preamble
%%%%%%%%%%%%%%%%%%%%%%%%%%%%%%%%%%%%%%%%%%%%%%%%%%%%%%%%%%%%%%%%%%%%%%%%%%%%%%%
\documentclass[12pt]{article}
\usepackage[letterpaper, margin=1in]{geometry}
\usepackage[version=4]{mhchem}
\usepackage{booktabs}
\usepackage{graphicx}
\usepackage{xcolor}
\usepackage{fancyhdr}
\usepackage{titlesec}
\usepackage{siunitx}
\usepackage{amsmath}
\usepackage{enumitem}
\usepackage{braket}
\usepackage{amsmath}
\usepackage{wrapfig}
\usepackage[font=footnotesize,labelfont=bf]{caption}
\definecolor{shiv_purple}{rgb}{0.6       ,  0.19607843,  0.8}
\definecolor{shiv_blue}{rgb}{0.11764706,  0.56470588,  1.}
\definecolor{shiv_green}{rgb}{0.        ,  0.57647059,  0.23529412}
\definecolor{shiv_yellow}{rgb}{0.97647059,  0.75686275,  0.1372549}
\definecolor{shiv_orange}{rgb}{1.        ,  0.54901961,  0.}
\definecolor{shiv_red}{rgb}{0.93333333,  0.20784314,  0.18039216}
\definecolor{shiv_gray}{rgb}{0.72156863,  0.71764706,  0.73333333}
\usepackage[sectionbib,sort&compress,super]{natbib}
\usepackage{doi}
%%%%%%%%%%%%%%%%%%%%%%%%%%%%%%%%%%%%
% Section Size
%%%%%%%%%%%%%%%%%%%%%%%%%%%%%%%%%%%%
\titleformat*{\section}{\normalsize\scshape}
\titleformat*{\subsection}{\normalsize\itshape}
\titleformat*{\subsubsection}{\normalsize\itshape}
%%%%%%%%%%%%%%%%%%%%%%%%%%%%%%%%%%%%
% Sub Sections as Letters
%%%%%%%%%%%%%%%%%%%%%%%%%%%%%%%%%%%%
\renewcommand{\thesubsection}{\thesection.\alph{subsection}}

%%%%%%%%%%%%%%%%%%%%%%%%%%%%%%%%%%%%
% Header
%%%%%%%%%%%%%%%%%%%%%%%%%%%%%%%%%%%%
\pagestyle{fancy}
\fancyhf{}
\chead{Shiv Upadhyay}
\rhead{\today}
\lhead{HF notes}
\cfoot{Page \thepage}
%%%%%%%%%%%%%%%%%%%%%%%%%%%%%%%%%%%%
%%%%%%%%%%%%%%%%%%%%%%%%%%%%%%%%%%%%%%%%%%%%%%%%%%%%%%%%%%%%%%%%%%%%%%%%%%%%%%%
% Main
%%%%%%%%%%%%%%%%%%%%%%%%%%%%%%%%%%%%%%%%%%%%%%%%%%%%%%%%%%%%%%%%%%%%%%%%%%%%%%%
\title{HF notes}
\date{\today}
\author{Shiv Upadhyay}
\begin{document}
\maketitle

Here, we will be preforming the Hartree-Fock self consistent field method for a water molecule with the cc-pVDZ basis. 
This method finds mean field solution to the electronic Hamiltonian. 

\section{Load Data}
\begin{enumerate}
\item Load the convergence criteria: the maximum iterations \texttt{data/iteration\_max.txt}, convergence of density matrix \texttt{data/convergence\_DM.txt}, and convergence of the energy \texttt{convergence\_E.txt}. The Hartree-Fock method is an iterative method and these decide when the method has converged. We'll talk more about these below.
\item Load in the integrals. These integrals are the overlap \texttt{data/S.txt}, the kinetic \texttt{data/T.txt}, the nuclear attraction \texttt{data/V.txt}, and the electron repulsion integrals \texttt{data/eri.txt}. These are real-valued integrals over gaussian fucntions.
\item Load in the nuclear repulsion energy \texttt{data/E\_nuc.txt}, the number of electrons alpha \texttt{data/num\_elec\_alpha.txt}, the number of electrons beta \texttt{data/num\_elec\_beta.txt}, and the number of atomic orbitals \texttt{data/num\_ao.txt}.
\end{enumerate}

\section{Set up}
Hartree-Fock is an iterative method, and these iterations are concluded when we reach some \textit{convergence criteria}. The convergence criteria we use  here are on the electronic energy and root mean squared change of the density matrix.

For the Hartree-Fock method, we need some starting guess. The easiest starting guess is to start the density matrix as all zeros. This is sometimes called the ``H Core'' guess since the core Hamiltonian is diagonalized at the first iteration.


\begin{enumerate}
\item Specifically, we will need to store: 
\begin{itemize}
\item iteration number \texttt{iteration\_num (int)}
\item current density matrix \texttt{D (array of doubles size [num\_ao, num\_ao])}
\item previous density matrix \texttt{D\_last (same type and size as D)}
\item current electronic energy \texttt{E\_elec (double)}
\item previous electronic energy \texttt{E\_elec\_last (double)}
\item electronic energy difference \texttt{iteration\_E\_diff (double)}
\item density matrix root mean squared difference \texttt{iteration\_rmsc\_dm (double)}
\item flags for convergence or exceeding iteration \texttt{converged (bool), exceeded\_iterations (bool)}
\end{itemize}

\item The HF eigenvalue problem $FC = ESC$ is a generalized eigenvalue problem. This is because we are working in a nonorthogonal atomic orbital basis. We can simply this to a standard eigenvalue problem $FC' = EC'$ by rotating to an orthogonal basis.
\end{enumerate}

\section{}
%%%%%%%%%%%%%%%%%%%%%%%%%%%%%%%%%%%%
% Citations
%%%%%%%%%%%%%%%%%%%%%%%%%%%%%%%%%%%%
%%%%%%%%%%%%%%%%%%%%%%%%%%%%%%%%%%%%%%%%%%%%%%%%%%%%%%%%%%%%%%%%%%%%%%%%%%%%%%%
% End
%%%%%%%%%%%%%%%%%%%%%%%%%%%%%%%%%%%%%%%%%%%%%%%%%%%%%%%%%%%%%%%%%%%%%%%%%%%%%%%
\end{document}
